\section{Vorwort}
\section{Kurzfassung}
\section{Abstract}

Variational autoencoders (VAEs) find application in multiple fields where they solve multiple problems like 
generative tasks or clustering in an unsupervised manner. This work strives to understand lower dimensional 
latent representations of remote sensing images produced by variational autoencoders. The purpose of this is that
the VAEs have learned to represent features of the remote sensing images in their latent space and therefore it
can be used for unsupervised clustering.
Code to reproduce the experiments is publicly available here:
\texttt{https://github.com/HannesStaerk/bachelorThesis}




%potential titles 
%Understanding Variational Autoencoder's Latent Representations of Remote Sensing Images
%Understanding the Latent Representation from Variational Autoencoders of Remote Sensing Images