\section{Architecture}
\subsection{Loss Function}

\paragraph{Entropy}\mbox{}\\
A part of the loss function used in the variational autoencoder is based on Kullback-Leibner divergence.
To understand Kullback-Leibler divergence it seems necessary to explain entropy from
the field of information theory. In short, entropy is a measure for the minimum average size an
encoding for a piece of information can possibly have.

Suppose there is a system $S1$ that can have four different states $a, b, c, d$
and every one of those states is equally likely to occur, that means the probability $P(x)$
of each state $x$ is $1/4$. Now the goal is to losslessly transmit all information about that system
with the minimum average amount of bits. That can be done with onyl two bits for example like this

\begin{center}
    \begin{tabular} {c c c c}
        $a: 00$ & $b: 01$ & $c: 10$ & $d: 11$
    \end{tabular}
\end{center}

However, if $P(a)=1$ and $P(b)=P(c)=P(d)=0,$ zero bits will suffice to encode the information since
it is always certain that the system is in state a. So the entropy of the system clearly depends on
the probablities of each state. To see in which way, one can consider the system $S2$ with 
$P(a)=1/2,\ P(b)=1/4,\ P(c)=1/8$ and $P(d)=1/8$. In that case it would be best to encode the state
with the highest probability with as few bits as possible since it has to be transmitted the most
often. That means $a$ is encoded with one bit as $0$. When decoding the information there must be
no ambiguities so while the encoding for $b$ has to start with a $1$ it cannot be $1$ since we need 
to encode two more states so $b: 10$. Additionally if $c: 11$ there would be no space left for $d$:
say $d: 111$ then if the transmitted information is $111111\dots$ it could either be decoded to
$ccc\dots$ or $dd\dots$. So $c$ should rather be encoded as $110$ which way $d: 111$ works. In the end a valid
encoding that can transmit all information with the minimum average amount of bits is

\begin{center}
    \begin{tabular} {c c c c}
        $a: 0$ & $b: 10$ & $c: 110$ & $d: 111$
    \end{tabular}
\end{center}

Here the states $c, d$ are encoded with three bits instead of the two bits in the first example.
But $c$ and $d$ are transmitted far less often than $a$ which now only needs one bit. To be more precise
half of all transmissions have one bit. Additionally a quarter of all transmissions have two bits. 
The sum of those probabilities multiplied with the respective amount of bits is the average amount of bits
needed to transfer the information in a given encoding. So in the example, with $f(x)$ as the number of
bits that encode a state $x$, that turns out to be $P(a)f(a)+P(b)f(b)+P(c)f(c)+P(d)f(d)=1.75$.
That means for $S2$ on average you only need $1.75$ bits to encode a state and since that is also the
minimum $1.75$ is the entropy of $S2$.

In general in an optimal encoding $f(x)$ is the same as $\log _{2} (\frac{1}{P(x)})$. For $S1$ $P(x)$ is $1/4$
so the number of bits for $x$ is $\log_{2} (4)=2$ what matches the two bits the first encoding uses 
for the states of $S1$.

The entropy of a system $S$ with a set of discrete events $X$ and the probability distribution $P(x)$
for each $x\in X$ is
\[ \sum_{x\in X} P(x)\log _{2} (\frac{1}{P(x)}) = -\sum_{x\in X} P(x)\log _{2} (P(x))\]
Intuitively if a system has high entropy, the size of the encodings are high on average and many states
have small probabilities. This means it is hard to predict what state the system will be in at a given time
since there is no state that can be guessed with high confidence. If entropy is low, zero for example,
one can be confident that the system is in a certain state like in the previous example with $P(a)=1$.

%wie muss hier der fall P(x) = 0 abgesichert werden weil log(0) und 1/0 ?

\paragraph{Cross Entropy} \mbox{}\\
