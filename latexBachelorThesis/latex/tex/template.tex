\section{Setup}

\subsection{Data}

The available data are 1024x1024 images of the two United States cities Jacksonville
in Florida and Omaha in Nebraska taken from the US3D Dataset that
was partially published to provide research data for the problem
of 3D reconstruction \parencite{2019-bosch-semantic}.
The images for each recorded area cover one square kilometer and can be divided 
into four categories with the first one being optical satellite images with three channels (RGB). 
Secondly visible and near infrared satellite images with eight channels (VNIR). 
Thirdly digital surface models (DSM). And lastly semantic labeling with five different categories.
\medskip

The optical images were taken by the WorldView-3 satellite of Digital Globe from 2014 to 2016
and contain seasonal and daily differences in vegetation and sun positions. 
Each pixel of an image is described by three bytes representing the intensity of either red, green or blue.

Also collected by WorldView-3 were the VNIR images which contain eight channels for eight different 
bands of the spectrum with a ground sample distance of 1.3 meters. These images were taken over the 
course of all twelve months making them usable for training models that can handle seasonal 
appearance differences which are even more distinct than in the RGB data because certain 
wavelengths capture shadows and vegetation especially well. Overall this data offers more 
detail than the three channel RBG pictures. The eight channels of the imagery correspond 
to the following wavelengths:

\begin{tabular} {c c}
    \parbox{5cm}{
        \begin{itemize}
            \item Coastal: 400 - 450 nm 			
            \item Blue: 450 - 510 nm			
            \item Green: 510 - 580 nm 			
            \item Yellow: 585 - 625 nm
        \end{itemize}
    }
    \parbox{5cm}{
        \begin{enumerate} 			
            \item Red: 630 - 690 nm
            \item Red Edge: 705 - 745 nm
            \item Near-IR1: 770 - 895 nm
            \item Near-IR2: 860 - 1040 nm
        \end{enumerate}
    }
\end{tabular}
\bigskip

The given DSMs were collected using light detection and ranging technology (Lidar). 
They have a single channel that describes the height of each pixel with a greater number 
representing a higher distance to the ground. 

Lastly there are semantic labeled pictures with one channel of a single byte encodes one of five 
different topographic classes. Those classes are vegetation, water, ground, building and clutter. 
The semantic labeling was done automatically from lidar data but manually checked and corrected afterwards.

For all four categories of data the area covered in a single image
is one square kilometer and they contain a lot of oblique view of 
buildings, often with sunshine casting good shadows making the 
data ideal for training models that should detect them.

\subsection{Experiments}

With knowledge about the provided Data we can start thinking about possible experiments to
gain insight into the 